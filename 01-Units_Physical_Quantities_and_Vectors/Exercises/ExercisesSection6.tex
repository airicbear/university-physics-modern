\documentclass[12pt]{article}
\usepackage[margin=1in]{geometry} 
\usepackage{amsmath,amsthm,amssymb,amsfonts,mathtools}
\usepackage{fancyhdr}

\newenvironment{problem}[2][]{
    \begin{trivlist}
        \item[
            {\bfseries #1}
            {\bfseries #2.}
        ]
}{\end{trivlist}}

\newcommand{\chaptertitle}{Chapter 1 Units, Physical Quantities, and Vectors}
\newcommand{\sectiontitle}{\textsc{Section 1.6 Estimates and Orders of Magnitude}}
\newcommand{\name}{\textsc{Eric Nguyen}}

\pagestyle{fancy}
\fancyhead[LO]{\sectiontitle}
\fancyhead[RO]{\textsc{\today} \,\name}
\cfoot{\thepage}
\setlength{\headheight}{15pt}

\newcommand{\solution}{\medskip\noindent\textbf{Solution:}}
\newcommand{\Part}[1]{\shortintertext{(#1)}}

% UNITS
\newcommand{\unit}[1]{\, \text{#1}}

\newcommand{\cm}{\unit{cm}}
\newcommand{\m}{\unit{m}}
\newcommand{\km}{\unit{km}}
\newcommand{\ft}{\unit{ft}}
\newcommand{\inch}{\unit{in.}}
\newcommand{\mi}{\unit{mi}}
\newcommand{\gcm}{\unit{g/cm}}
\newcommand{\mum}{\, \mu \text{m}}
\newcommand{\mm}{\unit{mm}}

\newcommand{\Liter}{\unit{L}}
\newcommand{\gallon}{\unit{gallon}}
\newcommand{\kg}{\unit{kg}}
\newcommand{\g}{\unit{g}}
\newcommand{\lb}{\unit{lb}}

\newcommand{\mps}{\unit{m/s}}
\newcommand{\mpg}{\unit{mpg}}
\newcommand{\kmL}{\unit{km/L}}

\newcommand{\y}{\unit{y}}
\newcommand{\mo}{\unit{mo}}
\newcommand{\ns}{\unit{ns}}
\newcommand{\s}{\unit{s}}
\newcommand{\gs}{\unit{gs}}
\newcommand{\days}{\unit{days}}
\newcommand{\Day}{\unit{day}}
\newcommand{\hours}{\unit{hrs}}
\newcommand{\hour}{\unit{hr}}
\newcommand{\minutes}{\unit{mins}}
\newcommand{\minute}{\unit{min}}

\newcommand{\ftpns}{\unit{ft/ns}}
\newcommand{\nspft}{\unit{ns/ft}}

\begin{document}

\begin{problem}{1.17}
A rather ordinary middle-aged man is in the hospital for a routine checkup. The nurse writes ``200'' on the patient's medical chart but forgets to include the units. Which of these quantities could the 200 plausibly represent? The patient's (a) mass in kilograms; (b) height in meters; (c) height in centimeters; (d) height in millimeters; (e) age in months.
\begin{align}
\Part{a}
200 \kg \left(\frac{2.205 \lb}{1 \kg}\right) &= 441.0 \lb; \text{No}
\Part{b}
200 \m \left(\frac{3.281 \ft}{1 \m}\right) &= 656.2 \ft; \text{No}
\Part{c}
200 \cm \left(\frac{1 \m}{100 \cm}\right) \left(\frac{3.281 \ft}{1 \m}\right) &= 6.562 \ft; \text{No}
\Part{d}
200 \mm \left(\frac{1 \m}{1000 \mm}\right) \left(\frac{3.281 \ft}{1 \m}\right) &= 0.6562 \ft; \text{No}
\Part{e}
200 \mo \left(\frac{1 \y}{12 \mo}\right) &= 16 \frac{2}{3} \y; \text{No}
\end{align}
\end{problem}

\begin{problem}{1.19}
How many times does a typical person blink her eyes in a lifetime?
\begin{align}
80 \y \left(\frac{365.24 \days}{1 \y}\right) \left(\frac{24 \hours}{1 \Day}\right) \left(\frac{60 \minutes}{1 \hour}\right) \left(\frac{60 \s}{1 \minute}\right) \left(\frac{1 \unit{blink}}{6 \s}\right) \approx 4 \times 10^8 \unit{blinks}
\end{align}
\end{problem}

\begin{problem}{1.21}
In the Wagner's opera \textit{Das Rheingold}, the goddess Freia is ransomed for a pile of gold just tall enough and wide enough to hide her from sight. Estimate the monetary value of this pile. The density of gold is 19.3 $\gcm^3$, and take its value to be about \$10 per gram.
\begin{align}
19.3 \gcm^3 \left(\frac{\$10}{1 \g}\right) \left(154 \cm\right)^3 \approx \$70 \unit{million}
\end{align}
\end{problem}

\begin{problem}{1.23}
You are using water to dilute small amounts of chemicals in the laboratory, drop by drop. How many drops of water are in a 1.0-L bottle? (\textit{Hint:} Start by estimating the diameter of a drop of water.)
\begin{align}
1.0 \Liter \left(\frac{1000 \cm^3}{1 \Liter}\right) \left(\frac{10 \mm^3}{1 \cm^3}\right) \left(\frac{1 \unit{drops}}{0.05 \mm^3}\right) = 2 \times 10^5 \unit{drops}
\end{align}
\end{problem}

\end{document}
