\documentclass[12pt]{article}
\usepackage[margin=1in]{geometry} 
\usepackage{amsmath,amsthm,amssymb,amsfonts,mathtools}
\usepackage{fancyhdr}

\newenvironment{problem}[2][]{
    \begin{trivlist}
        \item[
            {\bfseries #1}
            {\bfseries #2}
        ]
}{\end{trivlist}}

\newcommand{\chaptertitle}{Chapter 1 Units, Physical Quantities, and Vectors}
\newcommand{\sectiontitle}{\textsc{Section 1.9 Unit Vectors}}
\newcommand{\name}{\textsc{Eric Nguyen}}

\pagestyle{fancy}
\chead{\sectiontitle \hfill \textsc{\today} \hfill \name}
\cfoot{\thepage}
\setlength{\headheight}{15pt}

\newcommand{\solution}{\medskip\noindent\textbf{Solution:}}
\newcommand{\Part}[1]{\shortintertext{(#1)}}
\newcommand{\magnitude}[1]{\lVert #1 \rVert}
\newcommand{\Vector}[2]{\langle #1, #2 \rangle}
\newcommand{\UVector}[2]{\left(#1\right)\ihat + \left(#2\right)\jhat}
\newcommand{\ihat}{\hat{\imath}}
\newcommand{\jhat}{\hat{\jmath}}

% UNITS
\newcommand{\unit}[1]{\, \text{#1}}

\newcommand{\cm}{\unit{cm}}
\newcommand{\m}{\unit{m}}
\newcommand{\km}{\unit{km}}
\newcommand{\ft}{\unit{ft}}
\newcommand{\inch}{\unit{in.}}
\newcommand{\mi}{\unit{mi}}
\newcommand{\gcm}{\unit{g/cm}}
\newcommand{\mum}{\, \mu \text{m}}
\newcommand{\mm}{\unit{mm}}

\newcommand{\Liter}{\unit{L}}
\newcommand{\gallon}{\unit{gallon}}
\newcommand{\kg}{\unit{kg}}
\newcommand{\g}{\unit{g}}
\newcommand{\lb}{\unit{lb}}

\newcommand{\mps}{\unit{m/s}}
\newcommand{\mpg}{\unit{mpg}}
\newcommand{\kmL}{\unit{km/L}}

\newcommand{\y}{\unit{y}}
\newcommand{\mo}{\unit{mo}}
\newcommand{\ns}{\unit{ns}}
\newcommand{\s}{\unit{s}}
\newcommand{\gs}{\unit{gs}}
\newcommand{\days}{\unit{days}}
\newcommand{\Day}{\unit{day}}
\newcommand{\hours}{\unit{hrs}}
\newcommand{\hour}{\unit{hr}}
\newcommand{\minutes}{\unit{mins}}
\newcommand{\minute}{\unit{min}}

\newcommand{\ftpns}{\unit{ft/ns}}
\newcommand{\nspft}{\unit{ns/ft}}

\begin{document}

\begin{problem}{1.37}
Write each vector in Fig. E1.24 in terms of the unit vectors $\hat{\imath}$ and $\hat{\jmath}$.
\begin{align}
\vec{A} &= \UVector{0}{-8.00} \\
\vec{B} &= \UVector{7.50}{13.0} \\
\vec{C} &= \UVector{-10.9}{-5.07} \\
\vec{D} &= \UVector{-7.99}{6.02}
\end{align}
\end{problem}

\begin{problem}{1.39}
(a) Write each vector in Fig. E1.39 in terms of the unit vectors $\ihat$ and $\jhat$. (b) Use unit vectors to express vector $\vec{C}$, where $\vec{C} = 3.00 \vec{A} - 4.00 \vec{B}$. (c) Find the magnitude and direction of $\vec{C}$.
\begin{align}
\shortintertext{Find the components of $\vec{A}$ and $\vec{B}$:\medskip}
\vec{A}_x &= 3.60 \cos\left(70.0^\circ\right) \approx 1.23 \m \\
\vec{A}_y &= 3.60 \sin\left(70.0^\circ\right) \approx 3.38 \m \\
\vec{B}_x &= 2.4 \cos\left(30.0^\circ + 180^\circ\right) \approx -2.08 \m \\
\vec{B}_y &= 2.4 \sin\left(30.0^\circ + 180^\circ\right) \approx -1.20 \m
\Part{a}
\vec{A} &= \UVector{1.23}{3.38} \\
\vec{B} &= \UVector{-2.08}{-1.20}
\Part{b}
\vec{C} &= 3.00 \vec{A} - 4.00 \vec{B} \\
&= 3.00 \left(\UVector{1.23 \m}{3.38 \m}\right) - 4.00 \left(\UVector{-2.08 \m}{-1.20 \m}\right) \\
&= 3.69\ihat + 10.14\jhat + 8.32\ihat + 4.8\jhat \\
&\approx 12.0\ihat + 14.94\jhat
\Part{c}
\magnitude{\vec{C}} &= \sqrt{\left(12.0 \m\right)^2 + \left(14.94 \m\right)^2} \approx 19.2 \m \\
\angle\vec{C} &= \arctan{\left(\frac{14.94 \m}{12.0 \m}\right)} \times \frac{180}{\pi	} \approx 51.2^\circ
\end{align}
\end{problem}

\clearpage

\begin{problem}{1.41}
Given two vectors $\vec{A} = -2.00\ihat + 3.00\jhat + 4.00\hat{k}$ and $\vec{B} = 3.00\ihat + 1.00\jhat - 3.00\hat{k}$, (a) find the magnitude of each vector; (b) use unit vectors to write an expression for the vector difference $\vec{A} - \vec{B}$; and (c) find the magnitude of the vector difference $\vec{A} - \vec{B}$. Is this the same as the magnitude of $\vec{B} - \vec{A}$? Explain.
\begin{align}
\Part{a}
\magnitude{\vec{A}} &= \sqrt{\left(-2.00\right)^2 + \left(3.00\right)^2 + \left(4.00\right)^2} \approx 5.39 \\
\magnitude{\vec{B}} &= \sqrt{\left(3.00\right)^2 + \left(1.00\right)^2 + \left(-3.00\right)^2} \approx 4.36
\Part{b}
\vec{A} - \vec{B} &= \left(-2.00\ihat + 3.00\jhat + 4.00\hat{k}\right) - \left(3.00\ihat + 1.00\jhat - 3.00\hat{k}\right) \\
&= -5.00\ihat + 2.00\jhat + 7.00\hat{k}
\Part{c}
\magnitude{\vec{A} - \vec{B}} &= \sqrt{\left(-5.00\right)^2 + \left(2.00\right)^2 + \left(7.00\right)^2} = 8.83
\end{align}
Yes, this is the same as the magnitude of $\vec{B} - \vec{A}$ because every component is squared in the calculation of the magnitude, making any differences in signs irrelevant. 

\medskip

\noindent Verify:
\begin{align}
\vec{B} - \vec{A} &= \left(3.00\ihat + 1.00\jhat - 3.00\hat{k}\right) - \left(-2.00\ihat + 3.00\jhat + 4.00\hat{k}\right) \\
&= 5.00\ihat - 2.00\jhat - 7.00\hat{k} \\
\magnitude{\vec{B} - \vec{A}} &= \sqrt{\left(5.00\right)^2 + \left(-2.00\right)^2 + \left(-7.00\right)^2} = 8.83
\end{align}
\end{problem}

\end{document}
