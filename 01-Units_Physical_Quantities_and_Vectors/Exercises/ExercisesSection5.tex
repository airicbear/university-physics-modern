\documentclass[12pt]{article}
\usepackage[margin=1in]{geometry} 
\usepackage{amsmath,amsthm,amssymb,amsfonts,mathtools}
\usepackage{fancyhdr}

\newenvironment{problem}[2][]{
    \begin{trivlist}
        \item[
            {\bfseries #1}
            {\bfseries #2.}
        ]
}{\end{trivlist}}

\newcommand{\chaptertitle}{Chapter 1 Units, Physical Quantities, and Vectors}
\newcommand{\sectiontitle}{\textsc{Section 1.5 Uncertainty and Significant Figures}}
\newcommand{\name}{\textsc{Eric Nguyen}}

\pagestyle{fancy}
\fancyhead[LO]{\sectiontitle}
\fancyhead[RO]{\textsc{\today} \,\name}
\cfoot{\thepage}
\setlength{\headheight}{15pt}

\newcommand{\solution}{\medskip\noindent\textbf{Solution:}}
\newcommand{\Part}[1]{\shortintertext{(#1)}}

% UNITS
\newcommand{\unit}[1]{\, \text{#1}}

\newcommand{\cm}{\unit{cm}}
\newcommand{\m}{\unit{m}}
\newcommand{\km}{\unit{km}}
\newcommand{\ft}{\unit{ft}}
\newcommand{\inch}{\unit{in.}}
\newcommand{\mi}{\unit{mi}}
\newcommand{\gcm}{\unit{g/cm}}
\newcommand{\mum}{\, \mu \text{m}}
\newcommand{\mm}{\unit{mm}}

\newcommand{\Liter}{\unit{L}}
\newcommand{\gallon}{\unit{gallon}}
\newcommand{\kg}{\unit{kg}}
\newcommand{\g}{\unit{g}}

\newcommand{\mps}{\unit{m/s}}
\newcommand{\mpg}{\unit{mpg}}
\newcommand{\kmL}{\unit{km/L}}

\newcommand{\y}{\unit{y}}
\newcommand{\ns}{\unit{ns}}
\newcommand{\s}{\unit{s}}
\newcommand{\gs}{\unit{gs}}
\newcommand{\days}{\unit{days}}
\newcommand{\Day}{\unit{day}}
\newcommand{\hours}{\unit{hrs}}
\newcommand{\hour}{\unit{hr}}
\newcommand{\minutes}{\unit{mins}}
\newcommand{\minute}{\unit{min}}

\newcommand{\ftpns}{\unit{ft/ns}}
\newcommand{\nspft}{\unit{ns/ft}}

\begin{document}

\begin{problem}{1.14}
With a wooden ruler, you measure the length of a rectangular piece of sheet metal to be 12 mm.
With micrometer calipers, you measure the width of the rectangle to be 5.98 mm.
Use the correct number of significant figures:
What is (a) the area of the rectangle;
(b) the ratio of the rectangle’s width to its length;
(c) the perimeter of the rectangle;
(d) the difference between the length and the width;
and (e) the ratio of the length to the width?

\solution
\begin{align}
\Part{a}
A = 12 \mm \times 5.98 \mm &\approx 72 \mm^2
\Part{b}
5.98 \mm / 12 \mm &\approx 0.50 \mm
\Part{c}
P = 24 \mm + 11.96 \mm &\approx 36 \mm
\Part{d}
12 \mm - 5.98 \mm &\approx 6.0 \mm
\Part{e}
12 \mm / 5.98 \mm &\approx 2.0 \mm
\end{align}
\end{problem}

\begin{problem}{1.15}
A useful and easy-to-remember approximate value for number of seconds in a year is $\pi \times 10^7$. Determine the percent error in this approximate value. (There are 365.24 days in one year.)

\solution
\begin{align}
365.24 \days \left(\frac{24 \hours}{1 \Day}\right) \left(\frac{60 \minutes}{1 \hour}\right) \left(\frac{60 \s}{1 \minute}\right) &= 3.1556736 \times 10^7 \\
3.1556736 \times 10^7 - \pi \times 10^7 &= 140,809.46410206705 \\
\frac{140,809.46410206705}{3.1556736 \times 10^7} &= 0.004462104829284849 \\
&\approx 0.45\%
\end{align}
\end{problem}

\begin{problem}{1.16}
Express each approximation of $\pi$ to six significant figures:
(a) $22/7$ and (b) $355/113$. (c) Are these approximations accurate to that precision?

\solution
\begin{align}
\Part{a}
22/7 \approx 3.14286
\Part{b}
355/113 \approx 3.14159
\Part{c}
\text{These approximations are accurate to that precision.}
\end{align}
\end{problem}

\end{document}
