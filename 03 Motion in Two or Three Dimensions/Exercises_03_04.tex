\documentclass[12pt]{article}
\usepackage[margin=1in]{geometry} 
\usepackage{amsmath,amsthm,amssymb,amsfonts,mathtools}
\usepackage{fancyhdr}
\usepackage{graphicx}
\graphicspath{ {./figures/} }

\newenvironment{problem}[2][]{
    \begin{trivlist}
        \item[
            {\bfseries #1}
            {\bfseries #2}
        ]
}{\end{trivlist}}

\newcommand{\chaptertitle}{Chapter 3 Motion in Two or Three Dimensions}
\newcommand{\sectiontitle}{\textsc{3.4 Motion in a Circle}}
\newcommand{\name}{\textsc{Eric Nguyen}}

\pagestyle{fancy}
\chead{\sectiontitle \hfill \textsc{\today} \hfill \name}
\cfoot{\thepage}
\setlength{\headheight}{15pt}

\newcommand{\solution}{\medskip\noindent\textbf{Solution:}}
\newcommand{\Part}[1]{\shortintertext{(#1)}}
\newcommand{\PPart}[1]{\shortintertext{\qquad(#1)}}
\newcommand{\where}{, \, \text{ where }}
\newcommand{\magnitude}[1]{\lVert #1 \rVert}
\newcommand{\Vector}[2]{\langle #1, #2 \rangle}
\newcommand{\UVector}[2]{\left(#1\right)\ihat + \left(#2\right)\jhat}
\newcommand{\ihat}{\hat{\imath}}
\newcommand{\jhat}{\hat{\jmath}}
\newcommand{\radtodeg}[1]{\mbox{rad2deg}\left(#1\right)}

% UNITS
\newcommand{\unit}[1]{\, \text{#1}}

\newcommand{\cm}{\unit{cm}}
\newcommand{\m}{\unit{m}}
\newcommand{\km}{\unit{km}}
\newcommand{\ft}{\unit{ft}}
\newcommand{\inch}{\unit{in.}}
\newcommand{\mi}{\unit{mi}}
\newcommand{\gcm}{\unit{g/cm}}
\newcommand{\mum}{\, \mu \text{m}}
\newcommand{\mm}{\unit{mm}}
\newcommand{\rev}{\unit{rev}}

\newcommand{\Liter}{\unit{L}}
\newcommand{\gallon}{\unit{gallon}}
\newcommand{\kg}{\unit{kg}}
\newcommand{\g}{\unit{g}}
\newcommand{\lb}{\unit{lb}}

\newcommand{\mph}{\unit{mph}}
\newcommand{\kmh}{\unit{km/h}}
\newcommand{\kms}{\unit{km/s}}
\newcommand{\cms}{\unit{cm/s}}
\newcommand{\mps}{\unit{m/s}}
\newcommand{\mpg}{\unit{mpg}}
\newcommand{\kmL}{\unit{km/L}}
\newcommand{\rpm}{\unit{rpm}}
\newcommand{\rps}{\unit{rev/s}}

\newcommand{\y}{\unit{y}}
\newcommand{\mo}{\unit{mo}}
\newcommand{\ms}{\unit{ms}}
\newcommand{\ns}{\unit{ns}}
\newcommand{\s}{\unit{s}}
\newcommand{\gs}{\unit{gs}}
\newcommand{\days}{\unit{days}}
\newcommand{\Day}{\unit{day}}
\newcommand{\hours}{\unit{hrs}}
\newcommand{\hour}{\unit{hr}}
\newcommand{\Hour}{\unit{h}}
\newcommand{\minutes}{\unit{mins}}
\newcommand{\minute}{\unit{min}}

\newcommand{\ftpns}{\unit{ft/ns}}
\newcommand{\nspft}{\unit{ns/ft}}

\begin{document}

\begin{problem}{3.23}
    The earth has a radius of 6380 km and turns around once on its axis in 24 h.
    (a) What is the radial acceleration of an object at the earth's equator?
    Give your answer in m/s$^2$ and as a fraction of $g$.
    (b) If $a_{\text{rad}}$ at the equator is greater than $g$, objects will fly off the earth's surface and into space.
    (We will see the reason for this in Chapter 5.)
    What would the period of the earth's rotation have to be for this to occur?

    \solution
    \begin{align}
        \Part{a}
        R &= 6380 \km \left(\frac{1000 \m}{1 \km}\right) = 6\,380\,000 \m \\
        T &= 24 \Hour \left(\frac{60 \min}{1 \Hour}\right) \left(\frac{60 \s}{1 \min}\right) = 86\,400 \s \\
        a_{\text{rad}} &= \frac{4 \pi^2 R}{T^2} = \frac{4 \pi^2 \left(6\,380\,000 \m\right)}{\left(86\,400 \s\right)^2} \\
        &\approx 0.034 \mps^2 \text{ or } \approx 0.0034g
        \Part{b}
        9.80 \mps^2 &= \frac{4 \pi^2 \left(6\,380\,000 \m\right)}{T^2} \\
        T^2 &= \frac{4 \pi^2 \left(6\,380\,000 \m\right)}{9.80 \mps^2} \\
        T &= \sqrt{\frac{4 \pi^2 \left(6\,380\,000 \m\right)}{9.80 \mps^2}} \approx 5069.64 \s \\
        &= 5069.64 \s \left(\frac{1 \min}{60 \s}\right) \left(\frac{1 \Hour}{60 \min}\right) \approx 1.4 \Hour \\
    \end{align}
\end{problem}

\clearpage

\begin{problem}{3.25}
    \textbf{Pilot Blackout in a Power Dive.}
    A jet plane comes in for a downward dive as shown in \textbf{Fig. E3.25.}
    The bottom part of the path is a quarter circle with a radius of curvature of 280 m.
    According to medical tests, pilots will lose consciousness when they pull out of a dive at an upward acceleration greater than 5.5$g$. At what speed (in m/s and in mph) will the pilot black out during this dive?

    \solution
    \begin{align}
        a_{\text{rad}} &= \frac{4 \pi^2 R}{T^2} \\
        5.5g &= \frac{4 \pi^2 \left(280 \m\right)}{T^2} \\
        T^2 &= \frac{4 \pi^2 \left(280 \m\right)}{5.5 \left(9.80 \mps^2\right)} \\
        T &= \sqrt{\frac{4 \pi^2 \left(280 \m\right)}{5.5 \left(9.80 \mps^2\right)}} \approx 14.32 \s \\
        v &= \frac{2 \pi R}{T} = \frac{2 \pi \left(280 \m\right)}{14.32 \s} \approx 120 \mps \\
        v &= 120 \mps \left(\frac{3.281 \ft}{1 \m}\right) \left(\frac{1 \mi}{5280 \ft}\right) \left(\frac{60 \s}{1 \min}\right) \left(\frac{60 \min}{1 \Hour}\right) \approx 270 \mph
    \end{align}
\end{problem}

\begin{problem}{3.27}
    A Ferris wheel with radius 14.0 m is turning about a horizontal axis through its center (\textbf{Fig. E3.27}).
    The linear speed of a passenger on the rim is constant and equal to 6.00 m/s.
    What are the magnitude and direction of the passenger's acceleration as she passes through
    (a) the lowest point in her circular motion and
    (b) the highest point in her circular motion?
    (c) How much time does it take the Ferris wheel to make one revolution?

    \solution
    \begin{align}
        \Part{a}
        v &= \frac{2 \pi R}{T} \\
        6.00 \mps &= \frac{2 \pi \left(14.0 \m\right)}{T} \\
        T &= \frac{2 \pi \left(14.0 \m\right)}{6.00 \mps} = \frac{14 \pi}{3} \s \\
        a_{\text{rad}} &= \frac{4 \pi^2 R}{T^2} = \frac{4 \pi^2 \left(2.57 \m\right)}{\left(\frac{14 \pi}{3}\right)} \\
        &\approx 2.57 \mps^2 \text{ upward }
        \Part{b}
        &\approx 2.57 \mps^2 \text{ downward }
        \Part{c}
        T &= \frac{14 \pi}{3} \s \approx 14.7 \s
    \end{align}
\end{problem}

\begin{problem}{3.29}
    \textbf{Hypergravity.}
    At its Ames Research Center, NASA uses its large ``20-G'' centrifuge to test the effects of very large accelerations (``hypergravity'') on test pilots and astronauts.
    In this device, an arm 8.84 m long rotates about one end in a horizontal plane, and an astronaut is strapped in at the other end.
    Suppose that he is aligned along the centrifuge's arm with his head at the outermost end.
    The maximum sustained acceleration to which humans are subjected in this device is typically 12.5$g$.
    (a) How fast must the astronaut's head be moving to experience this maximum acceleration?
    (b) What is the \textit{difference} between the acceleration of his head and feet if the astronaut is 2.00 m tall?
    (c) How fast in rpm (rev/min) is the arm turning to produce the maximum sustained acceleration?

    \solution
    \begin{align}
        \Part{a}
        a_{\text{rad}} &= \frac{4 \pi^2 R}{T^2} \\
        12.5 g &= \frac{4 \pi^2 \left(8.84 \m\right)}{T^2} \\
        T^2 &= \frac{4 \pi^2 \left(8.84 \m\right)}{12.5 g} \\
        T &= \sqrt{\frac{4 \pi^2 \left(8.84 \m\right)}{12.5 g}} \approx 1.69 \s \\
        v &= \frac{2 \pi R}{T} = \frac{2 \pi \left(8.84 \mps\right)}{1.69 \s} \approx 32.9 \mps
        \Part{b}
        v_1 &= \frac{2 \pi R}{T} = \frac{2 \pi \left(10.84 \mps\right)}{1.69 \s} \approx 40.3 \mps \\
        a_{\text{rad}1} &= \frac{v^2}{R} = \frac{\left(40.3 \mps\right)^2}{10.84 \mps} \approx 150 \mps^2 \\
        a_{\text{rad}1} - a_{\text{rad}} &= \left(150 \mps^2\right) - 12.5 \left(9.80 \mps^2\right) \approx 27.5 \mps^2
        \Part{c}
        \rpm_{\text{max}} &= \frac{1 \rev}{1.69 \s} \left(\frac{60 \s}{1 \min}\right) \approx 35.5 \rpm
    \end{align}
\end{problem}

\end{document}

