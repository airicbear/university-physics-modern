\documentclass[12pt]{article}
\usepackage[margin=1in]{geometry} 
\usepackage{amsmath,amsthm,amssymb,amsfonts,mathtools}
\usepackage{fancyhdr}

\newenvironment{problem}[2][]{
    \begin{trivlist}
        \item[
            {\bfseries #1}
            {\bfseries #2}
        ]
}{\end{trivlist}}

\newcommand{\chaptertitle}{Chapter 2 Motion Along A Straight Line}
\newcommand{\sectiontitle}{\textsc{2.1 Displacement, Time, and Average Velocity}}
\newcommand{\name}{\textsc{Eric Nguyen}}

\pagestyle{fancy}
\chead{\sectiontitle \hfill \textsc{\today} \hfill \name}
\cfoot{\thepage}
\setlength{\headheight}{15pt}

\newcommand{\solution}{\medskip\noindent\textbf{Solution:}}
\newcommand{\Part}[1]{\shortintertext{(#1)}}
\newcommand{\magnitude}[1]{\lVert #1 \rVert}
\newcommand{\Vector}[2]{\langle #1, #2 \rangle}
\newcommand{\UVector}[2]{\left(#1\right)\ihat + \left(#2\right)\jhat}
\newcommand{\ihat}{\hat{\imath}}
\newcommand{\jhat}{\hat{\jmath}}

% UNITS
\newcommand{\unit}[1]{\, \text{#1}}

\newcommand{\cm}{\unit{cm}}
\newcommand{\m}{\unit{m}}
\newcommand{\km}{\unit{km}}
\newcommand{\ft}{\unit{ft}}
\newcommand{\inch}{\unit{in.}}
\newcommand{\mi}{\unit{mi}}
\newcommand{\gcm}{\unit{g/cm}}
\newcommand{\mum}{\, \mu \text{m}}
\newcommand{\mm}{\unit{mm}}

\newcommand{\Liter}{\unit{L}}
\newcommand{\gallon}{\unit{gallon}}
\newcommand{\kg}{\unit{kg}}
\newcommand{\g}{\unit{g}}
\newcommand{\lb}{\unit{lb}}

\newcommand{\kmh}{\unit{km/h}}
\newcommand{\mps}{\unit{m/s}}
\newcommand{\mpg}{\unit{mpg}}
\newcommand{\kmL}{\unit{km/L}}

\newcommand{\y}{\unit{y}}
\newcommand{\mo}{\unit{mo}}
\newcommand{\ns}{\unit{ns}}
\newcommand{\s}{\unit{s}}
\newcommand{\gs}{\unit{gs}}
\newcommand{\days}{\unit{days}}
\newcommand{\Day}{\unit{day}}
\newcommand{\hours}{\unit{hrs}}
\newcommand{\hour}{\unit{hr}}
\newcommand{\minutes}{\unit{mins}}
\newcommand{\minute}{\unit{min}}

\newcommand{\ftpns}{\unit{ft/ns}}
\newcommand{\nspft}{\unit{ns/ft}}

\begin{document}

\begin{problem}{2.1}
    A car travels in the $+x$-direction on a straight and level road. For the first 4.00 s of its motion, the average velocity of the car is $v_{\text{av}-x} = 6.25$ m/s. How far does the car travel in 4.00 s?
    \begin{align}
        v_{\text{av}-x} &= \frac{\Delta x}{\Delta t} = \frac{x_2 - x_1}{t_2 - t_1} \\
        6.25 \ms &= \frac{\Delta x}{4.00 \s - 0.00 \s} \\
        \Delta x &= 25.0 \m
    \end{align}
\end{problem}

\begin{problem}{2.3}
    \textbf{Trip Home.} You normally drive on the freeway between San Diego and Los Angeles at an average speed of 105 km/h (65 mi/h), and the trip takes 1 h and 50 min. On a Friday afternoon, however, heavy traffic slows you down and you drive the same distance at an average speed of only 70 km/h (43 mi/h). How much longer does the trip take?
    \begin{align}
        v_{\tex{av}-x} &= \frac{\Delta x}{\Delta t} = \frac{x_2 - x_1}{t_2 - t_1} \\
        105 \kmh &= \frac{\Delta x}{1 \unit{h} \, 50 \min} \\
        \Delta x &= \left(105 \kmh\right) \left(\frac{11}{6} \unit{h}\right) = 192.5 \km\\
        70 \kmh &= \frac{192.5 \km}{\Delta t} \\
        \Delta t &= \frac{192.5 \km}{70 \kmh} = 2.75 \unit{h} \\
        \Delta\left(\Delta t\right) &= \left(2.75 \unit{h}\right) \left(\frac{60 \min}{1 \unit{h}}\right) - \left(\frac{11}{6} \unit{h}\right)\left(\frac{60 \min}{1 \unit{h}}\right) = 55 \min
    \end{align}
\end{problem}

\begin{problem}{2.5}
    Starting from the front door of a ranch house, you walk 60.0 m due east to a windmill, turn around, and then slowly walk 40.0 m west to a bench, where you sit and watch sunrise. It takes you 28.0 s to walk from the house to the windmill and then 36.0 s to walk from the windmill to the bench, what are your (a) average velocity and (b) average speed?
    \begin{align}
        v_{\text{av}-x} &= \frac{\Delta x}{\Delta t} = \frac{x_2 - x_1}{t_2 - t_1} \\
        v_{\text{av}-x} &= \frac{20.0 \m}{28.0 \s + 36.0 \s} \approx 0.313 \mps \\
        s_{\text{av}-x} &= \frac{100.0 \m}{28.0 \s + 36.0 \s} \approx 1.56 \mps
    \end{align}
\end{problem}

\end{document}
