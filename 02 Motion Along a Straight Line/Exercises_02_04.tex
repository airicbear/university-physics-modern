\documentclass[12pt]{article}
\usepackage[margin=1in]{geometry} 
\usepackage{amsmath,amsthm,amssymb,amsfonts,mathtools}
\usepackage{fancyhdr}
\usepackage{graphicx}
\graphicspath{ {./figures/} }

\newenvironment{problem}[2][]{
    \begin{trivlist}
        \item[
            {\bfseries #1}
            {\bfseries #2}
        ]
}{\end{trivlist}}

\newcommand{\chaptertitle}{Chapter 2 Motion Along A Straight Line}
\newcommand{\sectiontitle}{\textsc{2.4 Motion with Constant Acceleration}}
\newcommand{\name}{\textsc{Eric Nguyen}}

\pagestyle{fancy}
\chead{\sectiontitle \hfill \textsc{\today} \hfill \name}
\cfoot{\thepage}
\setlength{\headheight}{15pt}

\newcommand{\solution}{\medskip\noindent\textbf{Solution:}}
\newcommand{\Part}[1]{\shortintertext{(#1)}}
\newcommand{\PPart}[1]{\shortintertext{\qquad(#1)}}
\newcommand{\where}{, \, \text{ where }}
\newcommand{\magnitude}[1]{\lVert #1 \rVert}
\newcommand{\Vector}[2]{\langle #1, #2 \rangle}
\newcommand{\UVector}[2]{\left(#1\right)\ihat + \left(#2\right)\jhat}
\newcommand{\ihat}{\hat{\imath}}
\newcommand{\jhat}{\hat{\jmath}}

% UNITS
\newcommand{\unit}[1]{\, \text{#1}}

\newcommand{\cm}{\unit{cm}}
\newcommand{\m}{\unit{m}}
\newcommand{\km}{\unit{km}}
\newcommand{\ft}{\unit{ft}}
\newcommand{\inch}{\unit{in.}}
\newcommand{\mi}{\unit{mi}}
\newcommand{\gcm}{\unit{g/cm}}
\newcommand{\mum}{\, \mu \text{m}}
\newcommand{\mm}{\unit{mm}}

\newcommand{\Liter}{\unit{L}}
\newcommand{\gallon}{\unit{gallon}}
\newcommand{\kg}{\unit{kg}}
\newcommand{\g}{\unit{g}}
\newcommand{\lb}{\unit{lb}}

\newcommand{\mph}{\unit{mi/h}}
\newcommand{\kmh}{\unit{km/h}}
\newcommand{\kms}{\unit{km/s}}
\newcommand{\cms}{\unit{cm/s}}
\newcommand{\mps}{\unit{m/s}}
\newcommand{\mpg}{\unit{mpg}}
\newcommand{\kmL}{\unit{km/L}}

\newcommand{\y}{\unit{y}}
\newcommand{\mo}{\unit{mo}}
\newcommand{\ms}{\unit{ms}}
\newcommand{\ns}{\unit{ns}}
\newcommand{\s}{\unit{s}}
\newcommand{\gs}{\unit{gs}}
\newcommand{\days}{\unit{days}}
\newcommand{\Day}{\unit{day}}
\newcommand{\hours}{\unit{hrs}}
\newcommand{\hour}{\unit{hr}}
\newcommand{\Hour}{\unit{h}}
\newcommand{\minutes}{\unit{mins}}
\newcommand{\minute}{\unit{min}}

\newcommand{\ftpns}{\unit{ft/ns}}
\newcommand{\nspft}{\unit{ns/ft}}

\begin{document}

\begin{problem}{2.19}
    An antelope moving with constant acceleration covers the distance between two points 70.0 m apart in 6.00 s.
    Its speed as it passes the second point is 15.0 m/s.
    What are
    (a) its speed at the first point and
    (b) its acceleration?

    \solution
    \begin{align}
        v_{\text{av-}x} &= \frac{x - x_0}{t} \\
        &= \frac{70.0 \m - 0 \m}{6.00 \s} = \frac{35 \m}{3 \s} \\
        &= \frac{1}{2} \left(v_{0x} + v_x\right) \\
        \Part{a}
        \frac{35}{3} \mps &= \frac{1}{2} \left(v_{0x} + 15.0 \mps\right) \\
        \frac{70}{3} \mps &= v_{0x} + 15.0 \mps \\
        v_{0x} &= \frac{25}{3} \mps \approx 8.33 \mps \\
        \Part{b}
        a_x &= \frac{v_x - v_{0x}}{t} \\
        &= \frac{15.0 \mps - \frac{25}{3} \mps}{6.0 \s} \\
        &= \frac{10}{9} \mps^2 \approx 1.11 \mps^2
    \end{align}
\end{problem}

\begin{problem}{2.21}
    \textbf{A Fast Pitch.} The fastest measured pitched baseball left the pitcher's hand at a speed of 45.0 m/s.
    If the pitcher was in contact with the ball over a distance of 1.50 m and produced constant acceleration,
    (a) what acceleration did he give the ball, and 
    (b) how much time did it take him to pitch it?

    \solution
    \begin{align}
        v_x^2 &= v_{0x}^2 + 2a_x \left(x - x_0\right) \\
        \left(45.0 \mps\right)^2 &= 0 + 2a_x \left(1.50 \m - 0\right) \\
        \Part{a}
        a_x &= \frac{2025 \m^2}{\s^2} \left(\frac{1}{2}\right) \left(\frac{1}{1.50 \m}\right) \\
        &= 675 \mps^2
        \Part{b}
        t &= \frac{v_x - v_{0x}}{a_x} \\
        &= \frac{45.0 \mps}{675 \mps^2} = \frac{3}{45} \s \approx 0.0667 \s
    \end{align}
\end{problem}

\clearpage

\begin{problem}{2.23}
    \textbf{Automobile Air Bags.} The human body can survive an acceleration trauma incident (sudden stop) if the magnitude of the acceleration is less than 250 m/s$^2$.
    If you are in an automobile accident with an initial speed of 105 km/h (65 mi/h) and are stopped by an airbag that inflates from the dashboard, over what distance must the airbag stop you for you to survive the crash?
    
    \solution
    \begin{align}
        v_{0x} &= 105 \kmh \left(\frac{1000 \m}{1 \km}\right) \left(\frac{1 \Hour}{60 \minute}\right) \left(\frac{1 \minute}{60 \s}\right) = \frac{175}{6} \mps \\
        v_x^2 &= v_{0x}^2 + 2a_x \left(x - x_0\right) \\
        \left(0 \mps\right)^2 &= \left(\frac{175}{6} \mps\right)^2 + 2 \left(-250 \mps^2\right) x \\
        x &= \frac{\left(\frac{175}{6} \mps\right)^2}{500 \mps^2} \approx 1.70 \m
    \end{align}
\end{problem}

\bigskip

\begin{problem}{2.25}
    \textbf{Air-Bag Injuries.} During an auto accident, the vehicle's air bags deploy and slow down the passengers more gently than if they had hit the windshield or steering wheel.
    According to safety standards, air bags produce a maximum acceleration of 60$g$ that lasts for only 36 ms (or less). How far (in meters) does a person travel in coming to a complete stop in 36 ms at a constant acceleration of 60$g$?

    \solution
    \begin{align}
        a_x &= 60g \left(\frac{9.81 \mps^2}{1 g}\right) = 588.6 \mps^2 \\
        t &= 36 \ms \left(\frac{1 \s}{1000 \ms}\right) = 0.036 \s \\
        x &= x_0 + v_{0x} t + \frac{1}{2} a_x t^2 \\
        &= \frac{1}{2} \left(588.6 \mps^2\right) \left(0.036 \s\right)^2 \approx 0.38 \m
    \end{align}
\end{problem}

\clearpage

\begin{problem}{2.27}
    \textbf{Are We Martians?} It has been suggested, and not facetiously, that life might have originated on Mars and been carried to earth when a meteor hit Mars and blasted pieces of rock (perhaps containing primitive life) free of the Martian surface.
    Astronomers know that many Martian rocks have come to the earth this way.
    (For instance, search the Internet for "ALH84001.")
    One objection to this idea is that microbes would have had to undergo an enormous lethal acceleration during the impact.
    Let us investigate how large such an acceleration might be.
    To escape Mars, rock fragments would have to reach its escape velocity of 5.0 km/s, and that would most likely happen over a distance of about 4.0 m during the meteor impact.
    (a) What would be the acceleration (in m/s$^2$ and $g$'s) of such a rock fragment, if the acceleration is constant?
    (b) How long would this acceleration last?
    (c) In tests, scientists have found that over 40\% of \textit{Bacillus subtilis} bacteria survived after an acceleration of 450,000$g$.
    In light of your answer to part (a), can we rule out the hypothesis that life might have been blasted from Mars to the earth?

    \solution
    \begin{align}
        v_x &= 5.0 \kms \left(\frac{1000 \m}{1 \km}\right) = 5000 \mps \\
        \Part{a}
        v_x^2 &= v_{0x}^2 + 2 a_x \left(x - x_0\right) \\
        \left(5000 \mps\right)^2 &= 2 a_x \left(4.0 \m\right) \\
        a_x &= \frac{\left(5000 \mps\right)^2}{8.0 \m} \approx 3.1 \times 10^6 \mps^2 \\
        a_x &= 3.1 \times 10^6 \mps^2 \left(\frac{1g}{9.81 \mps^2}\right) \approx 3.2 \times 10^5 g
        \Part{b}
        t &= \frac{v_x - v_{0x}}{a_x} = \frac{5000 \mps}{3.125 \times 10^6 \mps^2} = 0.0016 \s \\
        t &= 0.0016 \s \left(\frac{1000 \ms}{1 \s}\right) = 1.6 \ms
        \Part{c}
        \intertext{\indent No, we cannot rule out the hypothesis that life might have been blasted from Mars to the earth, considering that life can survive higher levels of acceleration than what is required to blast it from Mars to the earth. (i.e. $3.2 \times 10^5g < 4.5 \times 10^5g$)} \notag
    \end{align}
\end{problem}

\clearpage

\begin{problem}{2.29}
    At launch a rocket ship weights 4.5 million pounds.
    When it is launched from rest, it takes 8.00 s to reach 161 km/h;
    at the end of the first 1.00 min, its speed is 1610 km/h.
    (a) What is the average acceleration (in m/s$^2$) of the rocket
    (i) during the first 8.00 s and
    (ii) between 8.00 s and the end of the first 1.00 min?
    (b) Assuming the acceleration is constant during each time interval (but not necessarily the same in both intervals), what distance does the rocket travel
    (i) during the first 8.00 s and
    (ii) during the interval from 8.000 s to 1.00 min?

    \solution
    \begin{align}
         v_{60.0 \s} &= 1610 \kmh \left(\frac{1000 \m}{1 \km}\right) \left(\frac{1 \Hour}{60 \minute}\right) \left(\frac{1 \minute}{60 \s}\right) \approx 447 \mps \\
        v_{8.00 \s} &= 161 \kmh \left(\frac{1000 \m}{1 \km}\right) \left(\frac{1 \Hour}{60 \minute}\right) \left(\frac{1 \minute}{60 \s}\right) \approx 44.7 \mps \\
        \Part{a}
        \PPart{i}
        a_{\text{av-}x} &= \frac{v_{2x} - v_{1x}}{t_2 - t_1} = \frac{44.7 \mps}{8.00 \s} \approx 5.59 \mps^2 \\
        \PPart{ii}
        a_{\text{av-}x} &= \frac{v_{2x} - v_{1x}}{t_2 - t_1} = \frac{447 \mps - 44.7 \mps}{60.0 \s - 8.00 \s} \approx 7.74 \mps^2 \\
        \Part{b}
        \PPart{i}
        x &= x_0 + v_{0x} t + \frac{1}{2} a_x t^2 = \frac{1}{2} \left(5.59 \mps^2\right) \left(8.00 \s\right)^2 \approx 179 \m
        \PPart{ii}
        x &= x_0 + v_{0x} t + \frac{1}{2} a_x t^2 \\
        &= \left(44.7 \mps\right) \left(60.0 \s - 8.00 \s\right) + \frac{1}{2} \left(7.74 \mps^2\right) \left(60.0 \s - 8.00 \s\right)^2 \\ 
        &\approx 1.28 \times 10^4 \m
    \end{align}
\end{problem}

\clearpage

\begin{problem}{2.31}
    The graph in \textbf{Fig. E2.31} shows the velocity of a motorcycle police officer plotted as a function of time.
    (a) Find the instantaneous acceleration at $t = 3$ s, $t = 7$ s, and $t = 11$ s.
    (b) How far does the officer go in the first 5 s? The first 9 s? The first 13 s?
    
    \solution
    \begin{align}
        \Part{a}
        a_x &= \frac{v_x - v_{0x}}{t} = \frac{20 \mps - 20 \mps}{3 \s} = 0 \mps^2 \\
        a_x &= \frac{v_x - v_{0x}}{t} = \frac{45 \mps - 20 \mps}{9 \s - 5 \s} \approx 6.3 \mps^2 \\
        a_x &= \frac{v_x - v_{0x}}{t} = \frac{0 \mps - 45 \mps}{13 \s - 9 \s} \approx -11.3 \mps^2 
        \Part{b}
        x - x_0 &= \frac{1}{2} \left(v_{0x} + v_x\right) t \\
        x &= \frac{1}{2} \left(20 \mps + 20 \mps\right) \left(5 \s\right) = 100 \m \\
        x &= \frac{1}{2} \left(20 \mps + 45 \mps\right) \left(9 \s - 5 \s\right) + 100 \m = 230 \m \\
        x &= \frac{1}{2} \left(45 \mps + 0 \mps\right) \left(13 \s - 9 \s\right) + 230 \m = 320 \m 
    \end{align}
\end{problem}

\bigskip

\begin{problem}{2.33}
    A small block has a constant acceleration as it slides down a frictionless incline.
    The block is released from rest at the top of the incline, and its speed after it has traveled 6.80 m to the bottom of the incline is 3.80 m/s.
    What is the speed of the block when it is 3.40 m from the top of the incline?

    \solution
    \begin{align}
        v_x^2 &= v_{0x}^2 + 2a_x \left(x - x_0\right) \\
        \left(3.80 \mps\right)^2 &= 2a_x \left(6.80 \m\right) \\
        a_x &= \frac{14.44 \mps^2}{13.6 \m} \\
        v_x^2 &= v_{0x}^2 + 2a_x \left(x - x_0\right) \\
        v_x^2 &= 2 \left(\frac{14.44 \mps^2}{13.6 \m}\right) \left(3.40 \m\right) \\
        &= \sqrt{7.22 \mps^2} \approx 2.69 \mps 
    \end{align}
\end{problem}

\end{document}

